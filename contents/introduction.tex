% !TeX encoding = UTF-8
% !TeX root = ../main.tex

%% ------------------------------------------------------------------------
%% Copyright (C) 2021-2023 SJTUG
%% 
%% SJTUBeamer Example Document by SJTUG
%% 
%% SJTUBeamer Example Document is licensed under a
%% Creative Commons Attribution-NonCommercial-ShareAlike 4.0 International License.
%% 
%% You should have received a copy of the license along with this
%% work. If not, see <http://creativecommons.org/licenses/by-nc-sa/4.0/>.
%% -----------------------------------------------------------------------

\section{滑动窗口}

\subsection{定长滑动窗口}

\begin{frame}[fragile]{定长滑动窗口}
  \begin{columns}[T]
    \column{.8\textwidth}
    \begin{itemize}
      \item 基本套路:\textbf{入-更新-出}
            \begin{itemize}
              \item \textbf{入}:下标为 $i$ 的元素进入窗口,更新相关统计量。如果 $i<k−1$ 则重复第一步(\lstinline{continue})。
              \item \textbf{更新}:更新答案。一般是更新最大值/最小值。
              \item \textbf{出}:下标为 $i−k+1$ 的元素离开窗口,更新相关统计量。
            \end{itemize}
      \item 时间复杂度与空间复杂度
            \begin{itemize}
              \item 时间复杂度:$\mathcal{O}(n)$,其中 $n$ 为 $nums$ 的长度
              \item 空间复杂度:$\mathcal{O}(k)$
            \end{itemize}
    \end{itemize}
    \column{.2\textwidth}
    % \includegraphics[height=.4\textheight]{Knuth.jpg}

    % \includegraphics[height=.4\textheight]{Lamport.jpg}
  \end{columns}
\end{frame}


\begin{frame}[fragile]          % 注意添加 fragile 标记
  \frametitle{\textsc{2461. 长度为 K 子数组中的最大和}\link{https://leetcode.cn/problems/maximum-sum-of-distinct-subarrays-with-length-k}}
  % 代码块参数:语言,标题
  % 请减少代码初始的缩进
  \begin{codeblock}[language=python]{解答1}
class Solution:
  def maximumSubarraySum(self, nums: List[int], k: int) -> int:
    cnt = Counter(nums[:k - 1])
    ans, s = 0, sum(nums[:k - 1])
    for out, in_ in zip(nums, nums[k - 1:]):
        s += in_
        cnt[in_] += 1
        if len(cnt) == k:
            ans = max(ans, s)
        s -= out
        cnt[out] -= 1
        if cnt[out] == 0:
            del cnt[out]
    return ans
  \end{codeblock}
\end{frame}


\begin{frame}[fragile]          % 注意添加 fragile 标记
  % 代码块参数:语言,标题
  % 请减少代码初始的缩进
  \begin{codeblock}[language=python]{解答2}
class Solution:
  def maximumSubarraySum(self, nums: List[int], k: int) -> int:
      cnt = defaultdict(int)
      ans, s = 0, 0
      for i, num in enumerate(nums):
          cnt[num] += 1
          s += num
          if i < k - 1:
              continue
          if len(cnt) == k:
              ans = max(ans, s)
          s -= nums[i - k + 1]
          cnt[nums[i - k + 1]] -= 1
          if cnt[nums[i - k + 1]] == 0:
              del cnt[nums[i - k + 1]]
      return ans
  \end{codeblock}
\end{frame}

\begin{frame}[fragile]          % 注意添加 fragile 标记
  \frametitle{\textsc{2134. 最少交换次数来组合所有的 1 II}\link{https://leetcode.cn/problems/minimum-swaps-to-group-all-1s-together-ii}}
  % 代码块参数:语言,标题
  % 请减少代码初始的缩进
  \begin{codeblock}[language=python]{解答}
class Solution:
  def minSwaps(self, nums: List[int]) -> int:
    cnt1 = nums.count(1)
    nums *= 2
    # print(cnt1)
    # print(nums)
    ans = inf
    length = nums[:cnt1 - 1].count(1)
    for out, in_ in zip(nums, nums[cnt1 - 1:]):
        length += 1 if in_ == 1 else 0
        ans = min(ans, cnt1 - length)
        length -= 1 if out == 1 else 0
    return ans
  \end{codeblock}
\end{frame}


\begin{frame}[fragile]          % 注意添加 fragile 标记
  \frametitle{\textsc{30. 串联所有单词的子串}\link{https://leetcode.cn/problems/substring-with-concatenation-of-all-words}}
  % 代码块参数:语言,标题
  % 请减少代码初始的缩进
  \begin{codeblock}[language=python]{解答}
class Solution:
  def findSubstring(self, s: str, words: List[str]) -> List[int]:
    m, n = len(words), len(words[0])
    total_len = m * n
    cnt_words = Counter(words)
    cnt = Counter()
    ans = []
    for i in range(len(s) - total_len + 1):
        cnt = Counter([s[j:j + n] for j in range(i, i + total_len, n)])
        if cnt == cnt_words:
            ans.append(i)
    return ans
  \end{codeblock}
\end{frame}

\subsection{不定长滑动窗口}

\subsubsection{求最长/最大}

\begin{frame}[fragile]          % 注意添加 fragile 标记
  \frametitle{\textsc{3. 无重复字符的最长子串}\link{https://leetcode.cn/problems/longest-substring-without-repeating-characters}}
  % 代码块参数:语言,标题
  % 请减少代码初始的缩进
  \begin{codeblock}[language=python]{解答}
class Solution:
  def lengthOfLongestSubstring(self, s: str) -> int:
    ans = left = 0
    n = len(s)
    cnt = defaultdict(int)
    for right in range(n):
        cnt[s[right]] += 1
        while cnt[s[right]] > 1:
            cnt[s[left]] -= 1
            if cnt[s[left]] == 0:
                del cnt[s[left]]
            left += 1
        ans = max(ans, right - left + 1)
    return ans
  \end{codeblock}
\end{frame}


\begin{frame}[fragile]          % 注意添加 fragile 标记
  \frametitle{\textsc{2958. 最多 K 个重复元素的最长子数组}\link{https://leetcode.cn/problems/length-of-longest-subarray-with-at-most-k-frequency}}
  % 代码块参数:语言,标题
  % 请减少代码初始的缩进
  \begin{codeblock}[language=python]{解答}
class Solution:
  def maxSubarrayLength(self, nums: List[int], k: int) -> int:
    ans = left = 0
    n = len(nums)
    cnt = Counter()
    for right in range(n):
        cnt[nums[right]] += 1
        while cnt[nums[right]] > k:
            cnt[nums[left]] -= 1
            left += 1
        ans = max(ans, right - left + 1)
    return ans
  \end{codeblock}
\end{frame}

\begin{frame}[fragile]          % 注意添加 fragile 标记
  \frametitle{\textsc{2024. 考试的最大困扰度}\link{https://leetcode.cn/problems/maximize-the-confusion-of-an-exam}}
  % 代码块参数:语言,标题
  % 请减少代码初始的缩进
  \begin{codeblock}[language=python]{解答}
class Solution:
  def maxConsecutiveAnswers(self, answerKey: str, k: int) -> int:
    ans = left = 0
    cnt = Counter()
    n = len(answerKey)
    for right in range(n):
        cnt[answerKey[right]] += 1
        while cnt['T'] > k and cnt['F'] > k:
            cnt[answerKey[left]] -= 1
            left += 1
        ans = max(ans, right - left + 1)
    return ans
  \end{codeblock}
\end{frame}

\begin{frame}[fragile]          % 注意添加 fragile 标记
  \frametitle{\textsc{1004. 最大连续1的个数 III}\link{https://leetcode.cn/problems/max-consecutive-ones-iii}}
  % 代码块参数:语言,标题
  % 请减少代码初始的缩进
  \begin{codeblock}[language=python]{解答}
class Solution:
  def longestOnes(self, nums: List[int], k: int) -> int:
    ans = left = 0
    n = len(nums)
    zero = 0
    for right in range(n):
        zero += 1 if nums[right] == 0 else 0
        while zero > k:
            zero -= 1 if nums[left] == 0 else 0
            left += 1
        ans = max(ans, right - left + 1)
    return ans
  \end{codeblock}
\end{frame}

\begin{frame}[fragile]          % 注意添加 fragile 标记
  \frametitle{\textsc{1658. 将 x 减到 0 的最小操作数}\link{https://leetcode.cn/problems/minimum-operations-to-reduce-x-to-zero}}
  % 代码块参数:语言,标题
  % 请减少代码初始的缩进
  \begin{codeblock}[language=python]{解答}
class Solution:
  def minOperations(self, nums: List[int], x: int) -> int:
    target = sum(nums) - x
    if target < 0:
        return -1
    ans, left, s, n = -inf, 0, 0, len(nums)
    for right in range(n):
        s += nums[right]
        while s > target:
            s -= nums[left]
            left += 1
        if s == target:
            ans = max(ans, right - left + 1)
    return -1 if ans == -inf else n - ans
  \end{codeblock}
\end{frame}

\begin{frame}[fragile]          % 注意添加 fragile 标记
  \frametitle{\textsc{2516. 每种字符至少取 K 个}\link{https://leetcode.cn/problems/take-k-of-each-character-from-left-and-right}}
  % 代码块参数:语言,标题
  % 请减少代码初始的缩进
  \begin{codeblock}[language=python]{解答}
class Solution:
  def takeCharacters(self, s: str, k: int) -> int:
    cnt = Counter(s)
    targetA, targetB, targetC = cnt['a'] - k, cnt['b'] - k, cnt['c'] - k
    if targetA < 0 or targetB < 0 or targetC < 0:
        return -1
    ans, left, n, cnt = -inf, 0, len(s), Counter()
    for right in range(n):
        cnt[s[right]] += 1
        while cnt['a'] > targetA or cnt['b'] > targetB or cnt['c'] > targetC:
            cnt[s[left]] -= 1
            left += 1
        ans = max(ans, right - left + 1)
    return n - ans if ans != -inf else -1
  \end{codeblock}
\end{frame}


\begin{frame}[fragile]          % 注意添加 fragile 标记
  \frametitle{\textsc{2271. 毯子覆盖的最多白色砖块数}\link{https://leetcode.cn/problems/maximum-white-tiles-covered-by-a-carpet}}
  % 代码块参数:语言,标题
  % 请减少代码初始的缩进
  \begin{codeblock}[language=python]{解答}
class Solution:
  def maximumWhiteTiles(self, tiles: List[List[int]], carpetLen: int) -> int:
    tiles.sort(key = lambda x: x[0])
    ans, left, s, n = 0, 0, 0, len(tiles)
    for right in range(n):
        s += tiles[right][1] - tiles[right][0] + 1
        while tiles[right][1] - tiles[left][1] + 1 > carpetLen:
            s -= tiles[left][1] - tiles[left][0] + 1
            left += 1
        uncover = max(tiles[right][1] - tiles[left][0] - carpetLen + 1, 0)
        ans = max(ans, s - uncover)
    return ans
  \end{codeblock}
\end{frame}

\begin{frame}
  \frametitle{\centering 休息5分钟}
  \begin{center}
    \includegraphics[height=.4\textheight]{clock.jpg} % 请将图片文件名替换为实际的图片文件名
  \end{center}
  \printbibliography[heading=none]
\end{frame}

\subsubsection{求最短/最小}

\begin{frame}[fragile]          % 注意添加 fragile 标记
  \frametitle{\textsc{209. 长度最小的子数组}\link{https://leetcode.cn/problems/minimum-size-subarray-sum}}
  % 代码块参数:语言,标题
  % 请减少代码初始的缩进
  \begin{codeblock}[language=python]{解答}
class Solution:
  def minSubArrayLen(self, target: int, nums: List[int]) -> int:
    ans, left = inf, 0
    n, s = len(nums), 0
    for right in range(n):
        s += nums[right]
        while s >= target:
            ans = min(ans, right - left + 1)
            s -= nums[left]
            left += 1
    return ans if ans != inf else 0
  \end{codeblock}
\end{frame}

\begin{frame}[fragile]          % 注意添加 fragile 标记
  \frametitle{\textsc{1234. 替换子串得到平衡字符串}\link{https://leetcode.cn/problems/replace-the-substring-for-balanced-string}}
  % 代码块参数:语言,标题
  % 请减少代码初始的缩进
  \begin{codeblock}[language=python]{解答}
class Solution:
  def balancedString(self, s: str) -> int:
    m, n = len(s) // 4, len(s)
    cnt = Counter(s)
    if all(cnt[ch] == m for ch in 'QWER'):
        return 0
    ans, left = inf, 0
    for right, ch in enumerate(s):
        cnt[ch] -= 1
        while left <= right and all(cnt[ch] <= m for ch in 'QWER'):
            ans = min(ans, right - left + 1)
            cnt[s[left]] += 1
            left += 1
    return ans
  \end{codeblock}
\end{frame}

\begin{frame}[fragile]          % 注意添加 fragile 标记
  \frametitle{\textsc{76. 最小覆盖子串}\link{https://leetcode.cn/problems/minimum-window-substring}}
  % 代码块参数:语言,标题
  % 请减少代码初始的缩进
  \begin{codeblock}[language=python]{}
class Solution:
  def minWindow(self, s: str, t: str) -> str:
    left, min_len, n, cnt_s, cnt_t, ans = 0, inf, len(s), Counter(), Counter(t), ""
    for right, ch in enumerate(s):
        cnt_s[ch] += 1
        while cnt_s >= cnt_t:
            if min_len > right - left + 1:
                min_len = right - left + 1
                ans = s[left:right + 1]
            cnt_s[s[left]] -= 1
            if cnt_s[s[left]] == 0:
                del cnt_s[s[left]]
            left += 1
    return ans
  \end{codeblock}
\end{frame}


\begin{frame}[fragile]          % 注意添加 fragile 标记
  \frametitle{\textsc{2875. 无限数组的最短子数组}\link{https://leetcode.cn/problems/minimum-size-subarray-in-infinite-array}}
  % 代码块参数:语言,标题
  % 请减少代码初始的缩进
  \begin{codeblock}[language=python]{解答}
class Solution:
  def minSizeSubarray(self, nums: list[int], target: int) -> int:
    total, n = sum(nums), len(nums)
    nums *= 2
    ans = inf
    left = s = 0
    for right, num in enumerate(nums):
        s += num
        while s > target % total:
            s -= nums[left]
            left += 1
        if s == target % total:
            ans = min(ans, right - left + 1)
    return ans + target // total * n if ans < inf else -1
  \end{codeblock}
\end{frame}